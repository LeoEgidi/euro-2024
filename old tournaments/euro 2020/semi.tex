% Options for packages loaded elsewhere
\PassOptionsToPackage{unicode}{hyperref}
\PassOptionsToPackage{hyphens}{url}
%
\documentclass[
  10pt,
]{article}
\usepackage{lmodern}
\usepackage{amssymb,amsmath}
\usepackage{ifxetex,ifluatex}
\ifnum 0\ifxetex 1\fi\ifluatex 1\fi=0 % if pdftex
  \usepackage[T1]{fontenc}
  \usepackage[utf8]{inputenc}
  \usepackage{textcomp} % provide euro and other symbols
\else % if luatex or xetex
  \usepackage{unicode-math}
  \defaultfontfeatures{Scale=MatchLowercase}
  \defaultfontfeatures[\rmfamily]{Ligatures=TeX,Scale=1}
\fi
% Use upquote if available, for straight quotes in verbatim environments
\IfFileExists{upquote.sty}{\usepackage{upquote}}{}
\IfFileExists{microtype.sty}{% use microtype if available
  \usepackage[]{microtype}
  \UseMicrotypeSet[protrusion]{basicmath} % disable protrusion for tt fonts
}{}
\makeatletter
\@ifundefined{KOMAClassName}{% if non-KOMA class
  \IfFileExists{parskip.sty}{%
    \usepackage{parskip}
  }{% else
    \setlength{\parindent}{0pt}
    \setlength{\parskip}{6pt plus 2pt minus 1pt}}
}{% if KOMA class
  \KOMAoptions{parskip=half}}
\makeatother
\usepackage{xcolor}
\IfFileExists{xurl.sty}{\usepackage{xurl}}{} % add URL line breaks if available
\IfFileExists{bookmark.sty}{\usepackage{bookmark}}{\usepackage{hyperref}}
\hypersetup{
  pdftitle={EURO 2020 predictions: semi-finals},
  pdfauthor={Leonardo Egidi - DEAMS, University of Trieste, Italy. Mail: legidi@units.it},
  hidelinks,
  pdfcreator={LaTeX via pandoc}}
\urlstyle{same} % disable monospaced font for URLs
\usepackage[margin=1in]{geometry}
\usepackage{longtable,booktabs}
% Correct order of tables after \paragraph or \subparagraph
\usepackage{etoolbox}
\makeatletter
\patchcmd\longtable{\par}{\if@noskipsec\mbox{}\fi\par}{}{}
\makeatother
% Allow footnotes in longtable head/foot
\IfFileExists{footnotehyper.sty}{\usepackage{footnotehyper}}{\usepackage{footnote}}
\makesavenoteenv{longtable}
\usepackage{graphicx,grffile}
\makeatletter
\def\maxwidth{\ifdim\Gin@nat@width>\linewidth\linewidth\else\Gin@nat@width\fi}
\def\maxheight{\ifdim\Gin@nat@height>\textheight\textheight\else\Gin@nat@height\fi}
\makeatother
% Scale images if necessary, so that they will not overflow the page
% margins by default, and it is still possible to overwrite the defaults
% using explicit options in \includegraphics[width, height, ...]{}
\setkeys{Gin}{width=\maxwidth,height=\maxheight,keepaspectratio}
% Set default figure placement to htbp
\makeatletter
\def\fps@figure{htbp}
\makeatother
\setlength{\emergencystretch}{3em} % prevent overfull lines
\providecommand{\tightlist}{%
  \setlength{\itemsep}{0pt}\setlength{\parskip}{0pt}}
\setcounter{secnumdepth}{-\maxdimen} % remove section numbering
\usepackage{color}
\usepackage{bm}

\title{EURO 2020 predictions: semi-finals}
\author{Leonardo Egidi - DEAMS, University of Trieste, Italy. Mail:
\href{mailto:legidi@units.it}{\nolinkurl{legidi@units.it}}}
\date{4 July 2021}

\begin{document}
\maketitle

{
\setcounter{tocdepth}{2}
\tableofcontents
}
\hypertarget{the-statistical-model-in-brief}{%
\section{The statistical model (in
brief)}\label{the-statistical-model-in-brief}}

We use a \textbf{double Poisson model with dynamic team-specific
abilities} for the attack and the defence. Let \((X_{i}, Y_{i})\) denote
the random number of goals scored by the home and the away team in the
\(i\)-th game, \(i=1,\ldots,n\), respectively. \(\mathsf{ranking}\)
denotes the Coca-Cola FIFA ranking at May 27th, 2021, whereas att and
def denote the attack and the defence abilities, respectively.

\begin{align}
X_i| \lambda_{1i} &\sim \text{Poisson}(\lambda_{1i}),\\
Y_i|\lambda_{2i} &\sim \text{Poisson}(\lambda_{2i}),  \\
\log(\lambda_{1i}) &=\  \text{home} + \text{att}_{h_i, t}+ \text{def}_{a_i,t} + \frac{\gamma}{2}(\mathsf{ranking}_{h_i}-\mathsf{ranking}_{a_i}) \\
\log(\lambda_{2i}) & =\    \text{att}_{a_i,t} + \text{def}_{h_i,t} - \frac{\gamma}{2}(\mathsf{ranking}_{h_i}-\mathsf{ranking}_{a_i}), \ \ i=1,\ldots,n\ (\text{matches}), \\
\text{att}_{k, t} &\sim \ \mathcal{N}(\text{att}_{k, t-1}, \sigma^2), \\
\text{def}_{k, t} &\sim \  \mathcal{N}(\text{def}_{k, t-1}, \sigma^2),\\
\sum_{k=1}^{n_t} \text{att}_{k, }&=0, \  \sum_{k=1}^{n_t}\text{def}_{k, }=0, \ \ k=1,\ldots n_t \ (\text{teams}), \  t=1,\ldots, T \ (\text{times}).
\label{eq:scoring_rue}
\end{align}

Lines (1)-(2) display the likelihood's equations (two Poisson
distributions); lines (3)-(4) display the log-linear models for the
scoring rates \(\lambda_{1}, \lambda_{2}\); lines (5)-(6) display the
dynamic prior distributions for the attack and the defence parameters,
respectively; line (7) displays the sum-to-zero identifiability
constraints. Model fitting has been obtained through the Hamiltonian
Monte Carlo sampling, 2000 iterations, 4 chains (\texttt{rstan}
package). The historical data used to fit the models come from:
\textbf{Nations' League} (2019-2020), \textbf{Euro UEFA Qualifiers}
(2020-2021), \textbf{World Cup UEFA Qualifiers} (2021), \textbf{UEFA
Euro 2020} (groupstage + round of 16 + quarter of finals matches).

The idea is to provide a dynamic predictive scenario: at the end of each
match-day, the model will be refitted to predict the remaining matches.

\hypertarget{groupstage-predictions-semi-finals-6-7-july}{%
\section{Predictions: semi-finals (6-7
July)}\label{groupstage-predictions-semi-finals-6-7-july}}

Posterior matches probabilities from the posterior predictive
distribution of the model above are displayed in the table below.
\textbf{mlo} denotes the most likely exact outcome (in parenthesis, the
corresponding posterior probability). Darker regions in the plots below
denote more likely outcomes: on the \(x\)-axis the home goals, on the
\(y\)-axis the away goals.

\textcolor{red}{\textbf{Attention}}: \textbf{the matches probabilities below refer to the results
within the regular 90 minutes.}

\begin{longtable}[]{@{}llcccc@{}}
\toprule
home & away & home win & draw & away win & mlo\tabularnewline
\midrule
\endhead
Italy & Spain & 0.488 & 0.269 & 0.243 & 1-0 (0.139)\tabularnewline
England & Denmark & 0.505 & 0.269 & 0.226 & 1-0 (0.152)\tabularnewline
\bottomrule
\end{longtable}

\begin{center}\rule{0.5\linewidth}{0.5pt}\end{center}

\begin{center}\includegraphics[width=0.8\linewidth]{figs/data2-1} \end{center}

\hypertarget{expected-number-of-goals}{%
\section{Expected number of goals}\label{expected-number-of-goals}}

We compute also the \textbf{expected number of goals}
\(\lambda_1, \lambda_2\) for each match, obtained by computing the
median values from the MCMC sampling for the scoring rates.

\textcolor{red}{\textbf{Attention}}: \textbf{these expected goals do not represent the most likely results according to posterior probabilities.}

\begin{longtable}[]{@{}llcc@{}}
\toprule
home & away & exp\_home & exp\_away\tabularnewline
\midrule
\endhead
Italy & Spain & 1.36 & 0.85\tabularnewline
England & Denmark & 1.41 & 0.81\tabularnewline
\bottomrule
\end{longtable}

\newpage

\hypertarget{estimated-attackdefence-abilities}{%
\section{Estimated attack/defence
abilities}\label{estimated-attackdefence-abilities}}

In the plot below we display the posterior intervals for the
\textbf{attack} (red) and \textbf{defence} (blue) abilities estimated
through the training set matches, from \textbf{October 2019} until the
\textbf{round of 16}: the higher the attack and the lower the defence
values for a given team, and the better is the estimated overall team's
ability.



\begin{center}\rule{0.5\linewidth}{0.5pt}\end{center}

\begin{center}\includegraphics[width=0.9\linewidth]{figs/data-1} \end{center}

\hypertarget{euro-2020-winning-probabilities}{%
\section{Euro 2020 winning
probabilities}\label{euro-2020-winning-probabilities}}

We compute the final winning probabilities for each team. We simulated
in advance the two semi-finals and the final and we report the final
winning probabilities expressed in percentages (\%).

\textcolor{red}{\textbf{Attention}}: \textbf{these probabilities come from an ahead-simulation of the semi-finals and the final based on 4000 MCMC iterations.}

\begin{longtable}[]{@{}cc@{}}
\toprule
Winning team & Winning \%\tabularnewline
\midrule
\endhead
\textbf{Italy} & \textbf{34.2}\tabularnewline
England & 29.4\tabularnewline
Spain & 22.3\tabularnewline
Denmark & 14.0\tabularnewline
\bottomrule
\end{longtable}

\end{document}
